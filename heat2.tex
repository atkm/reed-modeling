\documentclass[11pt, fullpage]{article}
\usepackage{graphicx, , mathtools, latexsym, amsfonts, amsmath, setspace, stmaryrd,amssymb}
\usepackage[margin = 2.5cm]{geometry}
\def\Var{{\rm Var}\,}
\def\Cov{{\rm Cov}\,}
\newcommand{\qed}{\mbox{\rule{.07in}{.07in}}}
\newcommand{\rdb}{\right\rrbracket}
\newcommand{\ldb}{\left\llbracket}
\newcommand{\db}[1]{\ldb #1 \rdb}
\newcommand{\ssum}{\sum\limits}
\newcommand{\pprod}{\prod\limits}
\newcommand{\mc}[2]{\left({ #1 \choose #2 } \right) }
\newcommand{\s}{\sigma}
\newcommand{\E}{\exists\hspace{1mm}}
\newcommand{\Aut}{\text{Aut}}
\newcommand{\iso}{\overset\sim\to}
\newcommand{\Gal}{\text{Gal}}
\newcommand{\arrow}{\longrightarrow}
\newcommand{\Spl}{\text{Spl}}
\newcommand{\triq}{\trianglelefteq}
\newcommand{\Hom}{\text{Hom}}
\newcommand{\tr}{\text{tr}}
\begin{document}
\begin{flushright}
Wyatt Alt\\
1/31/2013\\
\end{flushright}
\begin{flushleft}
\textbf{MCM HEAT}\\
\end{flushleft}
This will be a two step problem.  First I'll solve heat for the batter with unknown boundary conditions on the sides, then I'll solve it for the pan to derive those boundary conditions.
\begin{enumerate}
\item[1.] Consider a rectangular mass of batter flush with the first octant and having width $W$, height $H$, and depth $l$.  Let the width of the baking pan be $D$.  Let $K$ be the conduction constant for brownie batter.  Let $T_i$ be the initial temperature of the batter and let $T_b$ be the baking temperature.  Let $f$ be the heat function for the brownie pan, let $r = (x,y,z)$ and $s = x+D,y+D,z+D$.   We wish to find the a function $T$ satisfying
\[\partial{T}{t} = K \nabla^2 T\]
subject to the initial condition
\[T(x,y,z,0) = T_i\]
and boundary conditions \begin{align*}
T(0,y,z,t) = T(W,y,z,t) &= f(s,t)\\
T(x,0,z,t) = T(x,L,z,t) &= f(s,t)\\
T(x,y,0,t) &= f(s,t)\\
T(x,y,H,t) &= T_b \end{align*}
where $s$ modifies the coordinates on the term to the left.
To simplify the following computations, define
\[u(r,t) = T(r,t) - f(s,t)\]
The equation becomes 
\[\frac{\partial u}{\partial t} K \nabla^2 u\]
with initial condition
\[u(r,0) = T_i - f(s,t)\]
and boundary conditions \begin{align*}
T(0,y,z,t) = T(W,y,z,t) &= 0\\
T(x,0,z,t) = T(x,L,z,t) &= 0\\
T(x,y,0,t) &= 0\\
T(x,y,H,t) &= T_b\end{align*}
We get away with not subtracting from the last equation because the top of the batter is not exposed to the pan.\\
\\
Approaching this by separation of variables, suppose
\[u(r,t) = XYZG\]
where $X,Y,Z$, and $G$ are functions of $x,y,z$, and $t$.  We find that \begin{align*}
u_t &= XYZG'\\
u_{xx} &= X''YZG\\
u_{yy} &= XY''ZG\\
u_{zz} &= XYZ''G \end{align*}
Plugging these into the heat equation and dividing by $KXYZG$, we find \begin{align*}
XYZG' &= K(X''YZG +XyY'ZG + XYZ''G)\\
\implies \frac{G'}{KG}  = \frac{X''}{X} + \frac{Y''}{Y} + \frac{Z''}{Z} \end{align*}
note that the left hand side of this equation is constant, so the right must be as well.  Introducing $\lambda^2$, we write
\[\frac{G'}{KG} = -\lambda^2 \hspace{15mm} \frac{X''}{X} + \frac{Y''}{Y} + \frac{Z''}{Z} = -\lambda^2\]
From the first equation, we immediately get the ODE
\[G'-G\lambda^2 = 0\]
For the other, let
\[\frac{X''}{X} = -\frac{Y''}{Y} - \frac{Z''}{Z} - \lambda^2 = -\mu^2\]
then 
\[\frac{Y''}{Y} = -\frac{Z''}{Z} - \lambda^2 + \mu^2 = -\nu^2\]
and
\[\frac{Z''}{Z} = - \lambda^2 + \mu^2 + \nu^2 = -\rho^2\]
\\
We can now derive the following equations: \begin{align*}
X''+ \mu^2 X &= 0\\
Y'' + \nu^2 Y &= 0\\
Z'' + \rho^2 Z &= 0 
\lambda^2 &= \mu^2 + \nu^2 + \rho^2
\intertext{and the one from before}
G'-G\lambda^2 = 0 \end{align*}
The general solution $G$ is $G = A e^{-\lambda^2 K t}$ for some constant $A$.  The general solution for $X'' + \mu^2 X = 0$ is 
\[X = c_1 \cos \mu x + c_2 \sin \mu x\]
Considering the boundary condition $X(0) = 0$, note that $c_1 = 0$.  Considering $X(W) = 0$, we must have $c_2 \sin \mu W = 0$.  Since $c_2=0$ is a trivial case, we focus on cases where $\sin \mu W = 0$.  This happens when $\mu W$ is an integer multiple of $\pi$.  We note therefore that
\[\mu_m = \frac{m \pi}{W}\]
for $m = 1,2, \dots$.  Substituting these findings into the equation for $X$, we find solutions
\[X_m(x) = \sin(\frac{m \pi x}{W})\]
the same method leads to a solution for $Y$:
\[Y_n(y) = \sin(\frac{n \pi y}{L})\]
for $n = 1,2,\dots$.\\
For $Z$, the boundary conditions are $Z(0) = 0$ and $Z(H) = T_b$ so we let
\[Z = \gamma_l e^{\rho_l z} + \delta_l e^{-\rho_l z}\]
the boundary condition $Z(0) = 0$ gives us $\gamma_l = -\delta_l$.  Choosing $\gamma_l = \frac{1}{2}$ we find
\[Z_l(z) = \sinh \rho_l z\]
the separated solution for $u$ is thus
\[u_{mnl}(r,t) = A_{mnl} \sin(\mu_m x) \sin (\nu_n y) \sinh (\rho_l z) e^{-\lambda^2_{mnl}Kt}\]
with
\[\lambda_{mnl}^2 = \mu_m^2 + \nu_n^2 + \rho_l^2 = \left(\frac{m \pi}{W}\right)^2 + \left(\frac{n \pi}{L} \right)^2 + \left(\frac{l \pi}{H}\right)^2\]
for $m,n,l = 1,2 \dots$\\
We can also form a linear superposition:
\[u(r,t) = \ssum_{m=1}^\infty \ssum_{n=1}^\infty \ssum_{l=1}^\infty A_{mnl} \sin(\mu_m x) \sin (\nu_n y) \sinh (\rho_l z) e^{-\lambda^2_{mnl}Kt}\]
with constants $A_{mnl}$.  This general solution satisfies the boundary conditions on all sides except the top.  To determine the values of the constants $A_{mnl}$, note the initial condition
\[u(r,0) = T_i - f(s,0)\]
we find
\[T_i - f(s,0) = \ssum_{m,=1}^\infty \ssum_{n=1}^\infty \ssum_{l=1}^\infty A_{mnl} \sin(\mu_m x) \sin (\nu_n y) \sinh (\rho_l z) \]
Let 
\[b_m(y,z) = \ssum_{n=1}^\infty \ssum_{l=1}^\infty A_{mnl} \sin (\nu_n y) \sinh (\rho_l z)\]
and consider that 
\[T_i - f(s,0) = \ssum_{m=1}^\infty b_m(y,z) \sin (\mu_m x)\]
we multiply both sides by $\sin (\mu_k x)$ and integrate, interchanging the integral and sum on the left: \begin{align*}
\int_0^W (T_i - f(s,0)) \sin(\mu_k x) dx &= \ssum_{m=1}^\infty \int_0^W b_m(y,z) \sin(\mu_m x) \sin (\mu_k x) dx
\intertext{Take note that on the right hand side, the sines are orthogonal unless $k = m$, leaving}
\int_0^W (T_i - f(s,0)) \sin(\mu_m x) dx &= \int_0^W b_m(y,z) \sin(\mu_m x) \sin (\mu_k x)
&= \frac{b_m(y,z) W}{2}\\
\end{align*}
and so we have 
\[b_m(y,z) = \frac{2}{W} \int_0^W (T_i - f(s,0)) \sin(\mu_m x) dx\]
the same technique will produce the analogous
\[c_n(x,z) = \frac{2}{L} \int_0^L (T_i - f(s,0)) \sin(\mu_n y) dy\]
for the opposite side. In the vertical direction, we let
\[d_l(x,y) = \ssum_{m=1}^\infty \ssum_{n=1}^\infty \sin(\mu_m x) \sin (\nu n y)\]
and so \begin{align*}
T_i - f(s,0) &= \ssum_{l=1}^\infty d_l(x,y) \sinh (\rho_l z)\\
&= \ssum_{l=1}^\infty d_l(x,y) \dfrac{e^{\rho_l z} - e^{-\rho_l z}}{2}\\
&= \frac{1}{2} \ssum_{l=1}^\infty d_l(x,y) (e^{\rho_l z} - e^{-\rho_l z})\\
2(T_i - f(s,0)) &= \ssum_{l=1}^\infty d_l (x,y) e^{\rho_l z} - \ssum_{l=1}^\infty d_l(x,y) e^{-\rho_l z} \end{align*}
let 
\[\delta(r) = \ssum_{l=1}^\infty d_l(x,y) e^{\rho_l z}\]
and 
\[\gamma(r) = \ssum_{l=1}^\infty d_l(x,y) e^{-\rho_l z}\]
Then
\[2(T_i - f(s,0)) = \delta(r) - \gamma(r)\]
and by fourier series,
\[d_n = \frac{2}{H} \int_0^H \delta(r) e^{-\rho_l z} dz = \frac{2}{H} \int_0^H \gamma(r) e^{\rho_l z} dz\]

\item[2] We used the method of finite differences to model the diffusion of heat through the batter and pan.  We started by building a two dimensional model for a vertical cross section of a rectangular pan.  Let $U^t$ and $C$ be arrays of dimension $M \times N$.  Let $c_a$,$c_p$ and $c_b$ be the thermal conductivity constants of air, the pan material, and brownie batter.  Let $T_i$ and $T_b$ stand for initial batter temperature and baking temperature, let $\delta s = \frac{1}{N}$, and let $w$ be the thickness of the pan material in units.\\
\\
From a schematic perspective, $U^t_{i,j}$ will hold the temperature of point $(i,j)$ at time $t$ and $C_{i,j}$ will hold the corresponding thermal conductivity constant.  For an illustrative example of initial conditions, consider the $6 \times 6$ case with $w = 1$:
\[U^0 = \begin{pmatrix} T_b&T_b&T_b&T_b&T_b&T_b\\
					T_b&T_i &T_i&T_i&T_i&T_b\\
					T_b&T_i &T_i&T_i&T_i&T_b\\
					T_b&T_i &T_i&T_i&T_i&T_b\\
					T_b&T_i &T_i&T_i&T_i&T_b\\
					T_b&T_b&T_b&T_b&T_b&T_b \end{pmatrix} \hspace{20mm}
  C = \begin{pmatrix} c_a&c_a&c_a&c_a&c_a&c_a\\
				   c_a &c_p &c_b&c_b&c_p&c_a\\
				    c_a &c_p &c_b&c_b&c_p&c_a\\
  				 c_a &c_p &c_b&c_b&c_p&c_a\\
				 c_a &c_p &c_p&c_p&c_p&c_a\\
				c_a&c_a&c_a&c_a&c_a&c_a \end{pmatrix} \]		
To track the evolution of $U$ through time, we define the following partial derivatives for points $(i,j,k)$, with time-coordinate $q$: \begin{align*}
u_t &\approx \dfrac{u_{i,j}^{q+1} - u_{i,j}^q}{\Delta t}\\
u_{xx} &\approx \dfrac{u_{i+1,j}^q - 2u^q{i,j} + u_{i-1,j}^q}{(\Delta s)^2}\\
u_{yy} &\approx \dfrac{u_{i,j+1}^q - 2u_{i,j}^q + u_{i,j-1}^q}{(\Delta s)^2} \end{align*}
Given these, the heat equation
\[u_t = c (u_{xx} + u_{yy})\]
becomes
\[\dfrac{u_{i,j}^{q+1} - u_{i,j}^q}{\Delta t} = c  \left(\dfrac{u_{i+1,j}^q - 2u_{i,j}^q + u_{i-1,j}^q}{(\Delta s)^2} + \dfrac{u_{i,j+1}^q - 2u_{i,j}^q + u_{i,j-1}^q}{(\Delta s)^2}\right)\]
from which we can derive
\[u_{i,j}^{q+1} = u_{i,j}^q + c \cdot \frac{\Delta t}{(\Delta s)^2} \left(u_{i+1,j}^q + u_{i-1,j}^q - 4u_{i,j}^q + u_{i,j+1}^q + u_{i,j-1}^q \right)\]
$\Delta t$, the length of a timestep, can be chosen arbitrarily but must obey the relation
\[\Delta t \leq \frac{(\Delta s)^2}{2c}\]
for $c = c_a,c_p,c_b$ for a stable solution.\\
\\
Using values $m = 8, n=8, T_b = 176, T_i = 26, \Delta t = 1/50000, c_a = .024$, and $c_b = .017$, , we varied the conduction coefficient for the pan to determine the best pan material for even heating around the edges.  We first compared glass ($c_p = 1.05$) to a much less conductive material ($c_p = 0.05$) and aluminum ($c_p = 215$) and reached the following states after 20000 timesteps:
\[c_p = 0.05: \begin{pmatrix} 176&176&176&176&176&176\\
						176&115.87&63.98&63.98&115.87&176\\
						176&93.66&38.64&38.64&93.66&176\\
						176&95.14&42.41&42.41&95.14&176\\
						176&122.43&95.38&95.38&122.43&176\\
					        176&176&176&176&176&176 \end{pmatrix}\] 
\[c_p = 1.05: \begin{pmatrix} 176&176&176&176&176&176\\
						176&136.34&71.99&71.99&136.34&176\\
						176&122.18&49.61&49.61&122.18&176\\
						176&127.61&60.92&60.92&127.61&176\\
						176&152.16&129.42&129.42&152.56&176\\
					        176&176&176&176&176&176 \end{pmatrix}\]
\[c_p = 215: \begin{pmatrix} 176&176&176&176&176&176\\
						176&136.85&72.48&72.48&136.85&176\\
						176&122.91&50.37&50.37&122.91&176\\
						176&128.41&62.09&62.09&128.41&176\\
						176&152.67&130.25&130.25&152.66&176\\
					        176&176&176&176&176&176 \end{pmatrix}\]
Noting the corner values of $122.43, 152.16$, and $152.67$, we were struck by the lack of difference between glass and aluminum.  We repeated the test on glass and aluminum with $14 \times 14$ arrays, and reached the following states in the lower left corners after 50,000 timesteps:
\[c_p = 1.05 \begin{pmatrix} \vdots&\vdots&\vdots&\vdots&\ddots  We concluded from this that, assuming a minimal level of conductivity (on par with glass), an increase in conductivity does little to increase or allay heat buildup around the edges of the pan.\\
\\
Our next step was to design an analogous scheme for the three-dimensional problem.  Using the partial derivatives \begin{align*}
u_t &\approx \dfrac{u_{i,j,k}^{k+1} - u_{i,j,k}^k}{\Delta t}\\
u_{xx} &\approx \dfrac{u_{i+1,j,k}^k - 2u^k{i,j,k} + u_{i-1,j,k}^k}{(\Delta s)^2}\\
u_{yy} &\approx \dfrac{u_{i,j+1}^k - 2u_{i,j}^k + u_{i,j-1}^k}{(\Delta s)^2}
u_{zz} &\approx \dfrac{u_{i,j+1}^k - 2u_{i,j}^k + u_{i,j-1}^k}{(\Delta s)^2}
\end{align*}
\end{enumerate}
\end{document}