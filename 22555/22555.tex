% A template for MCM contest solution
% Make sure 'reedmcm.cls' and 'mcm.sty' are in the directory that you're running latex from.
% These are meant to be used with pdflatex.
% Note: natbib for bibliography.
\documentclass[12pt]{reedmcm}
\usepackage{mcm}

%Title
\title{\textbf{Title of the Solution}}
\team{22555} % Put your team control number here. No actual names!
\contest{MCM}
\question{A} % Problem A or B?
\date{\today}

%Headers
\usepackage{fancyhdr}
\pagestyle{myheadings}
\setlength{\headheight}{13.6pt}
\setlength{\headsep}{20pt}
\pagestyle{fancy}
%%% Redefine the plain pagestyle so that it includes the ``page x of
%%% y'' information, as well.
\fancypagestyle{plain}{%
  \fancyhf{} % clear all header and footer fields
  \fancyhead[LO,RE]{Team \# 22555}
  \fancyhead[RO,LE]{Page~\thepage\ of \pageref{LastPage}}
  \fancyfoot[c]{\thepage}
  \renewcommand{\headrulewidth}{0pt}
  \renewcommand{\footrulewidth}{0pt}
}
%%% Define the header as per the contest regulations.
\fancyhead{}
\fancyhead[LO,RE]{Team \# 22555}
\fancyhead[RO,LE]{Page~\thepage\ of \pageref{LastPage}}


\begin{document}

% The Summary of your solution must be on the first page.
\begin{summary}
  The contest rules specify that you should include a one-page summary of your report. 
  You want a brief restatement of the problem followed by a largely \emph{non-technical} description of what you've done.
  Try to avoid using mathematical notation.
  You probably want to write a few paragraphs, around half to two-thirds of a page.

  From COMAP: 
  "The summary is a very important part of your MCM paper. The
    judges place considerable weight on the summary, and winning
    papers are sometimes distinguished from other papers based on
    the quality of the summary. To write a good summary, imagine
    that a reader may choose whether to read the body of the paper
    based on your summary. Thus, a summary should clearly describe
    your approach to the problem and, most prominently, what your
    most important conclusions were. The summary should inspire a
    reader to learn the details of your work.  Your concise
    presentation of the summary should inspire a reader to learn
    the details of your work. Summaries that are mere restatements
    of the contest problem, or are a cut-and-paste boilerplate
    from the Introduction are generally considered to be
    weak."

    To Summarize:
    \begin{description}
      \item[Restatement Clarification of the Problem]
        state in your own words what you are going to do.
      \item[Assumptions with Rationale/Justification]
        emphasize those assumptions that bear on the problem. List clearly all variables used in your model.
      \item[Model Design and justification for type model used/developed.]
      \item[Model Testing and Sensitivity Analysis, including error
        analysis, etc.]
      \item[Discuss strengths and weakness to your model or approach.] Include numerical results.
      \item[Provide algorithms] in words, figures, or flow charts (as a step by step algorithmic approach) for all computer codes developed.
    \end{description}
  \end{summary}
  % End of Summary

% After the Summary, the actual solution starts.
\maketitle
\tableofcontents
% \listoffigures
% \listoftables  

\section{Introduction}
I've outlined sections and subsections in this template, but there's no reason that you should follow the structure.
Read Cline's article for more.
Write an introduction to your report here. 
It should include a restatement of the problem, the history and context of the problem, and your work and results.
Mention the traditional methods that are used to solve the particular kind of problems.

some equation
\begin{equation*}
  \frac{−h^2}{2m} \frac{d^2\Phi}{dx^2} + kx^2\Phi = E\Phi
\end{equation*}

\section{The Model(s)}
\subsection{Assumptions}
Enumerate with roman numerals
\begin{enumerate}[(i)]
  \item un
  \item deux
  \item trois
\end{enumerate}
State the assumptions for your model.

\section{Solution 1}

\section{Solution 2}

\section{Testing and Results}

Example of tabular environment (using booktab environment. from wikibooks) 

\begin{tabular}{llr}
\toprule
\multicolumn{2}{c}{Item} \\
\cmidrule(r){1-2}
Animal    & Description & Price (\$) \\
\midrule
Gnat      & per gram    & 13.65      \\
          & each        & 0.01       \\
Gnu       & stuffed     & 92.50      \\
Emu       & stuffed     & 33.33      \\
Armadillo & frozen      & 8.99       \\
\bottomrule
\end{tabular}

\section{Conclusion}
blah blah\citep{knuth}

% The bibliography is in a separate file
\renewcommand{\bibname}{References}
\bibliographystyle{plainnat}
\nocite{*}
\bibliography{mcmbib}

\end{document}
