\documentclass[12pt]{article}
\usepackage{../mcm}

\begin{document}
\section{Shapes}

\subsection{Assumptions}
By the symmetry of the oven and positions of the heat sources, we may assume that an ideal shape must have the symmetry of some dihedral group.
We begin with exploring shapes that have the dihedral group of order 8, $D_4$.

\subsection{Arnold's Cat Map}
We use Arnold's Map (commonly called "Arnold's Cat Map") to produce exotic shapes.
Arnold's cat map is a chaotic, area-preserving map on a two-dimensional torus.
\citep{hilborn}
Area-preserving map is a measure-preserving map in a two-dimensional sense, i.e. $m(L^{-1}) = m(L)$, where $m$ is the Lebesgue measure.
For example, suppose $L$ is a linear transformation.
If $\det(L) = 1$ then it is an area-preserving map.
The Arnold's map $F$is defined as
\begin{equation*}
  F: (x,y) \mapsto (2x + y, x + y) \mbox{ (mod 1)}.
\end{equation*}
The corresponding matrix is
\begin{equation*}
A =
\begin{pmatrix}
    2 & 1  \\
    1 & 1  
  \end{pmatrix},
\end{equation*}
and clearly, $\det(A) = 1$.
The Arnold's cat map is an efficient way to search through the solution space of shapes of a fixed area.

\end{document}
