\documentclass[12pt]{article}
\usepackage{mcm}

\begin{document}
\section{Shapes}
\subsection{Area-preserving map}
Let L be a linear transformation.
If det(L) = 1 then it is an area-preserving map.
Suppose m ~ Lebesgue measure.
Area-preserving is equivalent to a measure-preserving map
i.e. $m(L^{-1}) = m(L)$.

\subsection{Arnold's Cat Map}
We use Arnold's Map (commonly called "Arnold's Cat Map") to produce exotic shapes.
Arnold's cat map is a chaotic, area-preserving map on a two-dimensional torus.
\citep{hilborn}
The Arnold's map $F$is defined as
\begin{equation*}
  F: (x,y) \mapsto (2x + y, x + y) \mbox{ (mod 1)}.
\end{equation*}
The corresponding matrix is
\begin{equation*}
A =
\begin{pmatrix}
    2 & 1  \\
    1 & 1  
  \end{pmatrix},
\end{equation*}
and clearly, $\det(A) = 1$.
The Arnold's cat map is an efficient way to search through the solution space of shapes of a fixed area.

\end{document}
